\documentclass{article}

\usepackage[T1]{fontenc}
\usepackage[polish]{babel}
\usepackage{float}
\usepackage[utf8]{inputenc}
\usepackage{graphicx}% Include figure files
\usepackage{dcolumn}% Align table columns on decimal point
\usepackage{bm}% bold math
\usepackage[cmex10]{amsmath}
\usepackage{polski}
\usepackage{geometry}


\author{Katarzyna Ladra, Mateusz Syc}		%autorzy sprawozdania
\title{Szablon sprawozdania w \LaTeX-u}		%temat wykonywanego ćwiczenia
\date{}
\begin{document}
\newgeometry{tmargin=2cm, bmargin=2cm, lmargin=2 cm, rmargin=2cm}
\maketitle %uwaga! tutaj automatycznie zostanie wpisana data wykonania sprawozdania


Data wykonania ćwiczenia: 24.04.2018 (Lab. 5)
%data wykonania ćwiczenia i numer laboratorium (1- pierwsze zajęcia laboratoryjne)
\\

\section{Wstęp}
W sprawozdaniu można umieścić krótki wstęp
wyjaśniający cel danego ćwiczenia i wprowadzający najważniejsze terminy, odwołanie do literatury \cite{USRP, Szostka, IQ, ieee}. Nie należy przekopiowywać instrukcji, ani internetu.

\section{Symulacja toru radiowego z modulacją BPSK}

\subsection{Tor nadawczy (liczba podrozdziałów przy analizie wyników i ich tytuły oczywiście będą się różnic dla poszczególnych ćwiczeń)}

Przedstawienie wyników symulacji zgodnie z instrukcją do danego ćwiczenia. Analiza uzyskanych wyników.



Wszystkie zrzuty z ekranu powinny być czytelne. Trzeba zadbać o odpowiednie wyskalowanie osi, ułożenie bloków itp. Wykresy zawsze powinny być opisane, zawierać legendę, opis osi. Każdy wykres powinien być przeanalizowany i podpisany. Rys. 1 przedstawia … \\

Probaaaaa!!!!

Dodatkowo przykład tabeli:

\begin{table}[!ht]
\label{tab:Static}
\centering
\begin{tabular}{ccccccc}
$t_{Ta}$ & MDL & $t_{CoFeB}$ & $\rho_{Ta}$ & $\mu_{0}M_{0}$ & $T_{C}$ & $\mu_{0}M_{S}$ \\
(nm) & (nm) & (nm) & ($\Omega\mu$cm) & (T) & (K) & (T) \\
\hline
5 & 0.55 & 0.36 & 217 & 0.93 & 472 & 0.50 \\
10 & 0.46 & 0.45 & 174 & 1.02 & 528 & 0.63 \\
15 & 0.39 & 0.52 & 171 & 1.11 & 650 & 0.80 \\
\end{tabular}
\end{table}

Przykład równania:

\begin{equation}
\xi_\mathrm{DL(FL)} = {\dfrac{2e}{\hbar}}\cdot {\dfrac{m}{A}}\cdot {\dfrac{\Delta H_\mathrm{L(T)}}{J_\mathrm{e}}}
\label{eq:efficiency}
\end{equation} \newpage



\section{Wnioski}

Najważniejsze wnioski i konkretne podsumowanie wyników. Bardzo cenne jest odniesienie do znanych systemów transmisji oraz do zagadnień poznanych na innych przedmiotach. \newline

Komentarz może być także w formie punktów.
\begin{enumerate}
\item pierwszy wniosek,
\item drugi,
\item trzeci,
\item etc.
\end{enumerate}

\bibliographystyle{unsrt}

\begin{thebibliography}{10}

\bibitem{USRP}
Nota katalogowa radia USRP \newline 
http://www.ni.com/datasheet/pdf/en/ds-355

\bibitem{Szostka}
J. Szóstka "Mikrofale" 2006

\bibitem{IQ}
Jakie jest Twoje IQ? Czyli o sygnałach i modulacjach kwadraturowych

http://mikrokontroler.pl/2014/11/26/sdr-jakie-jest-twoje-iq-czyli-o-sygnalach-i-modulacjach-kwadraturowych/

\bibitem{ieee}
Specyfikacje IEEE \newline http://www.radio-electronics.com/info/wireless/wi-fi/ieee-802-11a.php


\end{thebibliography}

\end{document}
